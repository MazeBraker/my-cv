%%%%%%%%%%%%%%%%%%%%%%%%%%%%%%%%%%%%%%%%%
% Medium Length Professional CV
% LaTeX Template
% Version 2.0 (8/5/13)
%
% This template has been downloaded from:
% http://www.LaTeXTemplates.com
%
% Original author:
% Rishi Shah 
%
% Important note:
% This template requires the resume.cls file to be in the same directory as the
% .tex file. The resume.cls file provides the resume style used for structuring the
% document.
%
%%%%%%%%%%%%%%%%%%%%%%%%%%%%%%%%%%%%%%%%%

%----------------------------------------------------------------------------------------
%	PACKAGES AND OTHER DOCUMENT CONFIGURATIONS
%----------------------------------------------------------------------------------------

\documentclass{resume} % Use the custom resume.cls style

\usepackage[left=0.75in,top=0.6in,right=0.75in,bottom=0.6in]{geometry} % Document margins
\usepackage{hyperref}
\usepackage[utf8]{inputenc}
\usepackage[russian]{babel}
\hypersetup{
    colorlinks=true,
    linkcolor=blue,
    filecolor=magenta, 
    urlcolor=blue,
}
\newcommand{\tab}[1]{\hspace{.2667\textwidth}\rlap{#1}}
\newcommand{\itab}[1]{\hspace{0em}\rlap{#1}}
\name{Горбачев Ринат} % Your name
\address{+7-977-751-99-01 | rgorbachev@edu.hse.ru}  % Your phone number and email
\address{\href{https://github.com/MazeBraker}{\underline{github: MazeBraker}} | \href{https://t.me/Rinaldinho1}{{\underline
{tg: Rinaldinho1}}}}

\begin{document}

\begin{rSection}{Умения}
\begin{tabular}{ @{} >{\bfseries}l @{\hspace{6ex}} l }
Языки программирования \ & Python, basics of C++\\
Фреймворки \ & Pandas, Numpy, Sklearn,  python-telegram-bot\\
Инструменты \ & Git, Jupyter, Linux, Latex, SQL\\
% Languages \ & Russian (native). English (Upper-Intermidiate). Kazakh(free to use) \\
% Hobbies \ & Chess, table tennis, football, jiu-jitsu
\end{tabular}

\end{rSection}

\begin{rSection}{Образование}
% {\bf Republican Physics and Mathematics School, Almaty} \hfill {\em September 2017 - June 2019} 
% \begin{itemize}
% \item Physics and Mathematics advanced profile. \hfill {\em GPA 4.3 / 5}
% \end{itemize} 
{\bf Высшая Школа Экономики, Москва(3й курс)} \hfill {\bf Сентябрь 2019 - Июнь 2023} 
\begin{itemize}
\item Бакалавриат: Прикладная математика и информатика
% \hfill {\em GPA 7 / 10}
\item Майнор: "Физика: От черных дыр к кубитам"
\end{itemize}

\end{rSection}

\begin{rSection}{Опыт работы}

{\bf Tesla Education - образовательный центр} \hfill {\bf Ноябрь 2020 - Декабрь 2021}
\begin{itemize}
\item Репетитор по предметам математика и программирование. Работа заключалась в составлении заданий, чтении лекций и подготовке к экзаменам
\end{itemize}
\end{rSection}

\begin{rSection}{Курсы}
% {\bf \href{https://github.com/MazeBraker/HSE_study}{\underline{University courses}}} \hfill {\em September 2019 - Present} 

{\bf Пройденые курсы}  \hfill {\bf Осень 2019 - Зима 2021} 
\begin{itemize}
\item Python, C++ , Calculus, Linear Algebra, Algorithms and Data Structures, Probability theory and
mathematical statistics
\end{itemize}
{\bf Текущие курсы}  \hfill {\bf Весна 2022 - Настоящее время} 
\begin{itemize}
\item Основные методы анализа данных, Прикладная статистика в машинном обучении, Введение в глубокое обучение, Теория баз данных и SQL
\end{itemize}
{\bf Факультативы}  \hfill {\bf Весна 2021 - Настоящее время} 
\begin{itemize}
\item Тинькофф: Анализ данных в продакшене, A/B тесты, Финтех тренды
\item Дойче Банк: Математические модели в инвестиционных банках
\end{itemize}
\end{rSection}



\begin{rSection}{Проекты}
{\bf \href{https://github.com/MazeBraker/HSE_study/tree/master/3_year}{\underline{Машинное обучение и другие проекты, связанные с анализом данных}}} \hfill {\bf Настоящее время}
\begin{itemize}
\item Были выполнены различные задачи анализа данных с использованием таких алгоритмов машинного обучения, как регрессия, кластеризация, градиентный спуск, деревья решений, бустинг
\end{itemize}
{\bf \href{https://github.com/MazeBraker/HSE_study/tree/master/3_year/Theory_of_databases}{\underline {E-commerce}}} \hfill {\bf Зима 2021} 
\begin{itemize}
\item Проект является финальным заданием курса по теории баз данных. В результате был написан магазин электронной комерции. Использовались различные технологии: Flask, HTML, DDL, SQL
\end{itemize}
% {\bf \href{https://github.com/MazeBraker/HSE_study/tree/master/2_year}{\underline {Educational practice: Data analysis via python}}} \hfill {\em Summer 2021}
% \begin{itemize}
% \item Build a parsing tool to get news/data from kg-portal Developed a telegram bot using open source libraries and my own developments \\
% Technologies used: Python, Numpy, pyTelegramBotAPI, Requests, Pandas, BeautifulSoup 
% \end{itemize}
{\bf \href{https://github.com/MazeBraker/HSE_study/tree/master/1_year/Linear_algebra}{\underline {Лабораторные работы по линейной регрессии, сжатию изображений}}} \hfill {\bf Весна 2019} 
\begin{itemize}
\item Были проведены многочисленные матричные вычисления, построены линейная и нелинейная регрессии с использованием MSE, сжатие изображения с помощью SVD. Используемые технологии: Numpy, Scipy, Matplotlib, Sklearn
\end{itemize}
\end{rSection}



% \begin{rSection}{Hobbies} 
% \begin{itemize}
% \item Chess, table tennis, football, jiu-jitsu
% \end{itemize}
% \end{rSection}

\end{document}